% ----------- Document settings -----------
\documentclass[nonacm,sigconf]{acmart}
% ====================================================
% General layout, typography & document-wide settings
% ====================================================

% ----------- Document settings -----------
\def\documentTitle      {Onderzoeken schrijven in Latex} % replace with actual document title
\def\documentSubtitle   {Een basis bestand voor een gelijke uitstraling} % replace with actual subtitle if any, else leave empty
\def\authorName         {John Doe} % replace with actual author name
\def\authorEmail        {JohnDoe@gmail.com} % replace with actual author email
\def\institutionName    {Harvard} % replace with actual institution name
\def\institutionCountry {Verenigde Staten} % replace with actual institution location
\def\institutionCity    {Massachusetts} % replace with actual institution location
\def\dueDate            {Oktober 2025} % replace with actual due date

\def\editorsVersion     {false} % true om editorsnote e.d. te tonen, false voor verbergen
\def\makeTitlePage      {true}  % true om titelpagina te maken, false om over te slaan
\def\makeTOCpage        {true}  % true om inhoudsopgave te maken, false om over te slaan

\def\listMark           {-} % itemize marker, e.g. '-', '*', '\textbullet', etc.

%----------- DONT TOUCH BELOW THIS LINE -----------
\usepackage{xparse}
\usepackage{xstring}

% ----------- Encoding & Fonts -----------
\usepackage[T1]{fontenc}
\usepackage[utf8]{inputenc}
\usepackage{scalefnt}

% ----------- Page layout -----------
\usepackage{expl3}
\usepackage{array}
\setlength{\columnsep}{0.333in}
\renewcommand{\baselinestretch}{1.05}
\setlength\parindent{2mm}
\setlength{\footskip}{40pt} %set footer size bigger to avoid overlap with text
\usepackage{geometry}
\geometry{
    a4paper,
    left=15mm,
    right=15mm,
    top=15mm,
    bottom=15mm,
}

% ----------- Table of Contents -----------
\usepackage{titletoc}
\setcounter{tocdepth}{2} % include subsections in TOC
%
 \titlecontents{section}
 [0] % i
 {\vspace{0.05cm}}
 {\thecontentslabel\enspace}%\thecontentslabel
 {}
 {\titlerule*[0.1cm]{.}\contentspage}%]

 \titlecontents{subsection}
 [1em] %
 {\vspace{0.05cm}}
 {\thecontentslabel\enspace}%\thecontentslabel
 {\hspace*{1em}}
 {\titlerule*[0.1cm]{.}\contentspage}


% ----------- Typography -----------
\usepackage{microtype} % subtle spacing & justification improvements

% ----------- Headings -----------
\usepackage{titlesec}
\titleformat{\section}{\large\bfseries}{\thesection}{1em}{}
\titleformat{\subsection}{\normalsize\bfseries}{\thesubsection}{0.75em}{}

% ----------- Lists -----------
\usepackage{enumitem}
\setlist[itemize]{
    leftmargin = *,
    listparindent = 10mm,
    label = {\listMark} % makes the default item marker a dash
}

% ----------- Figures & Tables -----------
\usepackage{graphicx}
\usepackage{subcaption}
\usepackage{tcolorbox}
\usepackage{float}
\usepackage{tabularx}
\usepackage{booktabs} % professional tables
\usepackage[table]{xcolor}
\usepackage{environ}

% Caption formatting
\usepackage{caption}
\captionsetup[table]{
    name=Tabel,                  % rename "Table" to "Tabel"
    labelfont={bf},              % "Tabel 1:" in bold
    textfont={normalfont},       % caption text in normal (non-bold)
    labelsep=colon               % ensures "Tabel 1:" (with colon)
}

\captionsetup[figure]{
    name=Figuur,                 % rename "Figure" to "Figuur"
    labelfont={bf},              % "Figuur 1:" in bold
    textfont={normalfont},       % caption text in normal (non-bold)
    labelsep=colon               % ensures "Figuur 1:" (with colon)
}

% ----------- Footnotes -----------
\usepackage[hang,flushmargin]{footmisc}

% ----------- Hyperlinks -----------
\usepackage[
    colorlinks=true,
    linkcolor=blue,
    citecolor=blue,
    urlcolor=blue,
]{hyperref}

% ----------- Column balance -----------
\usepackage{balance}

% ----------- Bibliography -----------
\usepackage[backend=biber,style=apa]{biblatex}
\DeclareLanguageMapping{dutch}{dutch-apa}
\DefineBibliographyStrings{dutch}{andothers = {et al.}}
\addbibresource{main.bib}
\setlength\bibitemsep{0.25em}

% ----------- ACM-style tweaks (optional) -----------
\settopmatter{printacmref=false}
\setcopyright{none}
\settopmatter{printfolios=true}
\renewcommand\footnotetextcopyrightpermission[1]{}

% ----------- Editors/general version -----------
\usepackage{comment}
\usepackage{ifthen}
\newboolean{editorsversion}
\setboolean{editorsversion}{\editorsVersion}  % change to false for student version


% ----------- Paper layout information -----------
\title{\documentTitle}
\subtitle{\documentSubtitle}

\author{\authorName}
\date{\dueDate}
\email{\authorEmail}
\affiliation{
    \institution{\institutionName}
    \city{\institutionCity}
    \country{\institutionCountry}
}

\hypersetup{
    pdfauthor={\authorName},
    pdftitle={\documentTitle},
    pdfborder={0 0 0}
}

% When \begin{document} is written, also add \maketitle and \tableofcontents there. (if settings are true)
\AtBeginDocument
{%
    \ifthenelse{\equal{\makeTitlePage}{true}}{%
        \maketitle
    }{}
    \ifthenelse{\equal{\makeTOCpage}{true}}{%
        \tableofcontents
        \newpage
    }{}
}% % import custom general paper settings
% --------------------------------
% This file contains a custom set
% of commands to make writing
% LaTeX easier
% --------------------------------

% makes 0.05 inches of horizontal space
\newcommand{\vertspace}{\vspace{0.05in}}

% writes centered italic text 80% width of the page/column and gives top & bottom padding
\newcommand{\question}[1]{\vertspace\begin{center}\parbox{0.8\linewidth}{\centering\textit{#1}}\end{center}\vertspace}

% Writes a single paragraph of Lorum Ipsum text
\newcommand{\lorem}{Lorem ipsum dolor sit amet, consectetur adipiscing elit. Nullam eget tortor a urna ornare pellentesque. Integer sit amet purus nec sem iaculis euismod. Duis at ipsum eu libero pharetra egestas. Quisque eleifend odio velit, at sollicitudin metus dictum eu. Integer nec mi congue, gravida nibh sed, faucibus mauris. Sed vel ipsum lobortis felis gravida dignissim. Curabitur vestibulum turpis eu orci lacinia consectetur.}

% APA style qoutes
\usepackage{xstring}
\newcommand{\qoute}[3]{%
  % Count the number of words in the quote
  \StrCount{#1}{ }[\woordenaantal]%
  \ifthenelse{\woordenaantal < 39}{``\textit{#1}'' (\citeauthor{#2}, \citeyear{#2}, p. #3)}% Kort citaat (minder dan 40 woorden)
  {
    \vertspace
    \begin{flushright}
      \parbox{0.95\linewidth}{\textit{#1}}
      (\citeauthor{#2}, \citeyear{#2}, p. #3)
    \end{flushright}
    \vertspace % Lang citaat (40 woorden of meer)
  }
}

% cites like \parencite[prenote][postnote]{cite}
\newcommand{\parencite}[3][]{%
    (#1 \citeauthor{#3}, \citeyear{#3}#2)%
}



% command for Editor's notes and comments
% defines custom commands:
% - \editorsonly{...} for inline comments
% - \begin{editorsonlyBox} ... \end{editorsonlyBox} for boxed comments
% - \editorsfootnote{...} for comments in footnotes
\ifthenelse{\boolean{editorsversion}}{%
    % EDITORS VERSION
    % inline red comments - compact, no paragraph indentation
    \newcommand{\editorsonly}[1]{%
        \par\vspace{0.05in}
        {\footnotesize\noindent\textcolor{red!70!black}{\textbf{Editor's note:} #1}}%
        \par\vspace{0.05in}
    }

    % environment for soft-background editor comments
    \newenvironment{editorsonlyBox}{%
        \par\medskip
        \begin{tcolorbox}[
            enhanced,
            breakable,
            colback=red!5,
            colframe=red!20!white,
            boxrule=0pt,
            borderline north={1pt}{0pt}{red!40!white},
            sharp corners,
            before skip=6pt,
            after skip=6pt,
            title={\textcolor{red!60!black}{\footnotesize Comment for editors:}},
            coltitle=red!70!black,
            fonttitle=\bfseries,
            top=2pt,
            bottom=2pt,
            left=4pt,
            right=4pt
        ]
    }{%
        \end{tcolorbox}
    }

    % red footnotes for editors
    \newcommand{\editorsfootnote}[1]{%
        \textcolor{red!70!black}{\footnote{\textcolor{red!70!black}{Editor's note: #1}}}%
    }

    % command to cite all references in the .bib file (for editors only)
    \nocite{*}
}{%
% STUDENT VERSION: suppress all editor content
    \newcommand{\editorsonly}[1]{\ignorespaces}
    \newenvironment{editorsonlyBox}{\comment}{\uncomment}
    \newcommand{\editorsfootnote}[1]{}
}

%%% ---------- TEXT SPLITTER ----------
%    The \splittext{<text>}{<n>} command splits <text> into lines of maximum <n> characters.
%    It inserts a line break (\\) every n characters.
%
%    Example usage:
%    \splittext{This is an example of splitting text into multiple lines every n characters.}{10}
%
%    This will produce:

\ExplSyntaxOn
\NewDocumentCommand{\spliteveryn}{mm}
 {
  % #1 = number of characters per line
  % #2 = text to split
  \str_set:Nn \l_tmpa_str { #2 }
  \int_zero:N \l_tmpa_int
  \str_map_inline:Nn \l_tmpa_str
   {
       ##1
       \int_incr:N \l_tmpa_int
       \int_compare:nNnT { \l_tmpa_int } = { #1 }
           {
           \\
           \int_zero:N \l_tmpa_int
       }
   }
 }
\ExplSyntaxOff

% --- Helper macro to check if argument is empty ---
\makeatletter
\newcommand{\IfNonEmpty}[2]{%
    \if\relax\detokenize{#1}\relax
    % empty → do nothing
    \else
    #2%
    \fi
}
\makeatother


%%% ---------- TABLE MACROS ----------
%    To use the SimpleTable command, use the following format:
%    \SimpleTable{<caption>}{<label>}{%
%        \TableRow{<Aspect 1>}{<Description 1>}
%        \TableRow{<Aspect 2>}{<Description 2>}
%        ...
%    }
%
%    Example usage:
%
%    \SimpleTable
%    {Overzicht van de belangrijkste elementen uit \textit{Smart Doorbell Security System Using IoT}.}
%    {tab:doorbell}
%    {
%        \TableRow{Probleemstelling}{Bestaande beveiligingssystemen vertrouwen enkel op gezichtsherkenning; onbekende gezichten worden niet gecontroleerd of gemeld.}
%        \TableRow{Doel van het systeem}{Ontwikkelen van een slimme deurbel die gebruikmaakt van IoT, gezichtsherkenning, stemherkenning en bewegingsdetectie om bezoekers automatisch te identificeren.}
%        \TableRow{Werking}{Bij het aanbellen activeert de Raspberry Pi een camera; de foto wordt vergeleken met een database van geregistreerde gezichten. Onbekende bezoekers leiden tot een e-mailmelding met foto en OTP naar de eigenaar.}
%        \TableRow{Technologieën}{Raspberry Pi, OpenCV (beeldverwerking), DSP (stemherkenning), Speech-to-Text, e-mailserver voor OTP-verzending.}
%    }
% --- Table Header + Row commands ---
\NewDocumentCommand{\TableHeader}{m m}{%
    \arrayrulecolor{black!80}\specialrule{1pt}{0pt}{2pt}
    \textbf{#1} & \textbf{#2} \\ \midrule
}
\NewDocumentCommand{\TableHeaderThree}{m m m}{%
    \arrayrulecolor{black!80}\specialrule{1pt}{0pt}{2pt}
    \textbf{#1} & \textbf{#2} & \textbf{#3} \\ \midrule
}

\NewDocumentCommand{\TableRow}{m m}{%
    \textbf{#1} & #2 \\ \arrayrulecolor{black!10}\specialrule{0.5pt}{1pt}{0pt}\arrayrulecolor{black!100}
}
\NewDocumentCommand{\TableRowThree}{m m m}{%
    #1 & #2 & #3 \\ \arrayrulecolor{black!10}\specialrule{0.5pt}{1pt}{0pt}\arrayrulecolor{black!100}
}


% --- Updated 2-column version ---
\newcommand{\SimpleTableTwo}[3]{%
    \begin{table}[h!]
        \centering
        \scalefont{0.75}%
        \resizebox{\columnwidth}{!}{%
            \begin{tabular}{@{}p{0.25\linewidth}p{0.70\linewidth}@{}}

                #3
                \arrayrulecolor{black!80}\specialrule{1pt}{0pt}{2pt}
            \end{tabular}%
        }%
        \scalefont{1}%
        \IfNonEmpty{#1}{\caption{#1}}%
        \IfNonEmpty{#2}{\label{#2}}%
    \end{table}%
}

% --- Updated 3-column version ---
\newcommand{\SimpleTableThree}[3]{%
    \begin{table}[h!]
        \centering
        \scalefont{0.75}%
        \resizebox{\columnwidth}{!}{%
            \begin{tabular}{@{}p{0.33\linewidth}p{0.33\linewidth}p{0.33\linewidth}@{}}
                \arrayrulecolor{black!80}\specialrule{1pt}{0pt}{2pt}
                #3
                \arrayrulecolor{black!80}\specialrule{1pt}{0pt}{2pt}
            \end{tabular}%
        }%
        \scalefont{1}%
        \IfNonEmpty{#1}{\caption{#1}}%
        \IfNonEmpty{#2}{\label{#2}}%
    \end{table}%
}

% --- Smart dispatcher: decides between 2 or 3 cols automatically ---
\NewDocumentCommand{\SimpleTable}{m m m}{%
    \IfSubStr{#3}{\TableHeaderThree}{%
        \SimpleTableThree{#1}{#2}{#3}%
    }{%
        \SimpleTableTwo{#1}{#2}{#3}%
    }%
}
 % import custom commands and packages

% ----------- Authors and title -----------
\title{De wereld buiten je voordeur}
\subtitle{Contextual integrity bij slimme deurbellen}

\author{Matt ter Steege}
\date{Oktober 2025} %due date
\email{m.j.ter.steege@students.uu.nl}
\affiliation{
    \institution{Universiteit Utrecht, 9932003}
    \city{Utrecht}
    \country{Nederland}
}

% ----------- Start document -----------

\begin{document}

    \maketitle

    \begin{editorsonlyBox}
    \begin{itemize}[leftmargin = *,listparindent =1cm]
        \item[X] Titel en eventuele subtitel, opleiding, naam schrijver, studentnummer schrijver.
        \item[-] Abstract
        \item[-] ‘Inleiding’ met kader, probleemstelling, onderzoeksvraag en aankondiging van de structuur van het verslag. Hierin verwerk je ook een stukje achtergrond met een beschrijving van de belangrijke concepten of ‘related work’.
        \item[-] Een ‘Methode’ met een beschrijving van de werkwijze die je gaat gebruiken om de onderzoeksvraag te beantwoorden.
        \item[-] Bespreking van de gevonden wetenschappelijke literatuur in een beschouwend geheel die aansluiten bij de onderzoeksvraag.
        \item[-] Een overkoepelende conclusie vanuit de verkregen inzichten en beantwoord de onderzoeksvraag met deze inzichten
    \end{itemize}
    \end{editorsonlyBox}

    \section{Introductie}
    We leven in een tijd waarin zoveel mogelijk onderdelen van iemands leven aan het internet gekoppeld (kunnen) worden.
    Zo ook je eigen voordeur: de opkomst van zogenaamde videodeurbellen, zoals deurbellen van Ring of Eufy, is een steeds bekender gezicht in de wijken van Nederland.
    Het plus- (en tevens ook min-)punt van deze producten is dat elke (verdachte) beweging die de deurbel detecteert, wordt opgenomen en doorgestuurd naar de eigenaar.
    Mogelijke inbrekers worden afgeschrikt door het idee dat ze op video staan bij een inbraakpoging en dat zorgt bij veel mensen voor een veilig gevoel, maar dit heeft ook een keerzijde.
    De postbode die je krantje komt bezorgen, maar ook voorbijrijdende auto’s, buren die een ommetje maken of kinderen die langsfietsen worden ook opgenomen, terwijl dit niet de doelgroep is waarvoor (of waartegen) deze deurbel ontworpen is.
    Dit roept de vraag op:

    \question{Hoe beïnvloedt het constant filmen van slimme deurbellen de privacy van buren en voorbijgangers?}

    Deze vraag sluit nauw aan bij het concept contextual integrity van Helen Nissenbaum, waarin iemand zo goed mogelijk in zijn of haar persoonlijke vrijheid gelaten wordt en data alleen in een passende context gedeeld mag worden.
    Videodeurbellen doorbreken deze verwachte informatiestromen, want waar voorbijgangers normaal anoniem over straat liepen, worden zij nu onbewust onderdeel van een digitaal surveillancesysteem.

    \subsection{Theoretisch kader}
    \begin{editorsonlyBox}
    CHECK: Contextual integrity uitleggen.\\
    CHECK: Benoem hoe dit verschilt van het klassieke idee van privacy (bijv. “control over information”).\\
    CHECK: Breng het naar jouw onderwerp: de context van de stoep voor een huis → normaal geen registratie, maar met een deurbelcamera wel. Hierdoor raakt de normale informatiestroom verstoord.
    \end{editorsonlyBox}

    \parencite{nissenbaum2009privacy}\editorsfootnote{Moet je nog lezen :(} Schreef al over een door haar ontwikkeld privacy theorie: \textbf{Contextual integrity}.
    Dit schreef zij in haar boek \textit{Privacy In Context: Technology, Policy, and the Integrity of Social Life.}

    \vertspace
    \begin{itemize}[leftmargin = *,listparindent =1cm]
        \item[-] Privacy wordt gewaarborgd door passende informatiestromen.
        \item[-] Passende informatiestromen zijn stromen die voldoen aan contextuele informatienormen.
        \item[-] Contextuele informatienormen verwijzen naar vijf onafhankelijke parameters: betrokkene, afzender, ontvanger, informatietype en transmissie-principe.
        \item[-] Concepties van privacy zijn gebaseerd op ethische overwegingen die in de loop der tijd evolueren.
    \end{itemize}
    \vertspace

    Nissenbaum stelt dat privacy en wat acceptabel is om te delen af hangt van de situatie waarin men op dat moment leeft.
    Een voorbeeld hiervan is dat het (doorgaans vaak) niet gewenst is om je medische dossier met Jan en alleman te delen, echter met een dokter of huisarts is dit natuurlijk wel wenselijk.
    Hier komt contextuele integriteit goed naar boven.
    Want gebaseerd op de situatie deel je (of wil je) wel of niet bepaalde data met bepaalde entiteiten en deze entiteiten deze data ook niet doorgeven aan andere waarvoor de data niet nodig is.

    Dit wijkt af van het "traditionele" denkbeeld, oftwel \textit{control over information} waarin een individu zelf zijn data beheert en kiest of data wel of niet gedeeld wordt.
    Dit is veel meer individu-gecentreerd en contextuele integriteit is meer (je raadt het al) context-gecentreerd.

    \subsection{Relevantie}
    Videodeurbellen zijn dus nauw verbonden met het concept contextuele integriteit.
    Voor eigen veiligheid (of gemoedsrust) schaffen steeds meer mensen een videodeurbel aan, dit gaat echter ten koste van de privacy van voorbijgangers, buren en andere die toevallig langs een huis met een videodeurbel lopen.
    Daarom wordt in dit onderzoek gekeken naar of de waarde in veiligheidsgevoel opweegt tegen het ongevraagd (en passief) filmen van voorbijgangers en dergelijke.

    \section{Methode}
    Dit onderzoek is een kwantitatief onderzoek om de vragen rondom de contextuele integriteit en privacy en Slimme (video) deurbellen te beantwoorden.
    Hiervoor is uitsluitend literatuuronderzoek gedaan.
    Er zijn veel verschillende bronnen geraadpleegd, dit is grotendeels via Google Scholar gedaan.
    Hierbij zijn verschillende zoektermen gebruikt zoals: Slimme deurbel, Ring (video)deurbel, Smart doorbell, Privacy video doorbell.

    De gebruikte bronnen zijn afkomstig uit wetenschappelijke publicaties en tijdschriften.
    Zo vormt het werk van Nissenbaum (2009) een theoretisch fundament op het gebied van privacy en contextual integrity.
    Artikelen van Shaffer (2021) en Tabassum & Lipford (2023) bouwen daarop voort met recente analyses van smart home-privacy en gebruikerscontrole, gepubliceerd in peer-reviewed journals.

    Daarnaast bieden studies van Liu (2021), Lalitha et al. (2019) en Chaudhari et al. (2020) een technisch perspectief op slimme deurbellen, waarbij veiligheid en functionaliteit empirisch worden onderzocht.
    Tot slot leveren Selinger & Durant (2022) en Kelly (2023) kritische beschouwingen over Amazon’s Ring en de maatschappelijke gevolgen van consumentgestuurde surveillance.
    Samen bieden deze bronnen een goed gebalanceerde mix van theoretische, technische en ethische invalshoeken, afkomstig uit betrouwbare en actuele academische contexten.

    \section{Beschouwing van literatuur}
    Hier ga je echt de gevonden artikelen samenbrengen in een doorlopend verhaal.
    Opdelen in subthema’s:

    \begin{itemize}[leftmargin = *,listparindent =1cm]
        \item[-] Bewegingsdetectie, cloudopslag, delen met politie.
        \item[-] Studies over hoe vaak mensen ongewild gefilmd worden; klachten; gevoelens van surveillance.
        \item[-] Wat bewoners ervaren: afschrikking, bewijs bij criminaliteit. Literatuur die laat zien of dit effect groot/klein is.
        \item[-] Hoe de informatiestromen door de deurbel afwijken van de ‘normale sociale verwachtingen’: een toevallige voorbijganger verwacht niet dat zijn route naar de supermarkt opgenomen en bewaard wordt.
        \item[-] Burenruzies, wantrouwen, normalisering van surveillance in de publieke ruimte.
    \end{itemize}

    \subsection{}
    \parencite{moh2023characterizing}


    -unauthorized use and missuse
    - unauthorized use means use without explicit permission. (alexa van een vriend vragen iets toe te voegen aan amazon cart <-> sensitive informatie opvragen als eigenaar niet thuis is)


    er zijn 10 categories van unautorized use:
    - entertainment; muziek, films e.d kijken via persoons account
    - Monitor activities: videocamera's bekijken zonder weten van gefilmde
    - Broken device: slopen van apparaat van eigenaar
    - Dataleakage: iemand krijgt toegang tot je data (payment information etc.)
    - Modify enviromet: ongevraagd aanpassen van dingen (temperatuur, tv settings, e.d.)
    - Modify data: liedjes toevoegen aan je playlist, volgorde van foto's aanpassen in digi-lijstje
    - Delete data: instellingen, presets, foto's e.d. verwijderen
    - Change settings: wat zou dat betekenen he..
    - Trigger unwanted behavior: alarm af laten gaan, alexa triggeren e.d.
    - Make purchase: heeft een aankoop gedaan zonder vragen via jouw account



    Dit is opgedeeld in 2 delen: Meegemaakt en gedaan
    Als we alleen kijken naar het "smart cameras" gedeelte, want de rest is niet heel relevant voor dit onderzoek. Dan komt de volgende data naar boven:

%    Zijn gemonitord:
%        \begin{tabular}{m{10cm}m{10cm}m{10cm}}
%            \hline
%            expliciet & impliciet & geen toestemming\\
%            \hline
%            $11+0+0 = 11$ & $11+0+4 = 15$ & $5+2+5 = 12$\\
%            \hline
%        \end{tabular}
%        \\

    \begin{table}[ht]

        \begin{adjustbox}{\textwidth}
        \begin{tabular}{|c|c|c|}
              \hline
            expliciet     & impliciet     & geen toestemming \\ \hline
            $11+0+0 = 11$ & $11+0+4 = 15$ & $5+2+5 = 12$     \\ \hline
        \end{tabular}
        \end{adjustbox}
        \caption{Aantal mensen die zijn of hebben gemonitord blablablablafsd sgfsdgfdhd  dsfhdfghf}
        \label{tab:table-1}
    \end{table}

    Er wordt in het onderzoek van Moh et al. naar 10 verschillende categoriën gekeken, maar die zijn voor dit onderzoek niet nuttig, die zijn dus buiten beschouwing gelaten en de tabel hierboven is de opsomming van de 3 categorien die wél nuttig waren.
    Deze categorien zijn: Monitor activities, data leakage & trigger unwanted behavior

    Hier beschouw je de literatuur: niet alleen “wat zegt studie X”, maar ook: hoe hangen deze bevindingen samen, waar spreken ze elkaar tegen, en wat valt jou op.

    \section{Conclusie}
    Duidelijk antwoord op je onderzoeksvraag.
    Trek de lijn terug naar contextual integrity: veiligheid en privacy staan niet los van elkaar, maar de balans verschuift zodra technologie te veel buiten de intended context gaat. (max 1 pagina)

    \section{Voorbeeldcitatie}
    Lamport beschreef het gebruik van als documentopmaaksysteem al in de jaren negentig \parencite{tabassum2023exploring}.
    Ook Knuths werk over blijft invloedrijk. \parencite{chaudhari2020smart}
    blablalbalbalba \parencite{kelly2023ring}

    \printbibliography

    \balance % balances last-page columns
\end{document}
