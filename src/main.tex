\documentclass[nonacm,sigconf]{acmart}

% ----------- General look -----------
\setlength{\columnsep}{0.333in} % tighter column gap
\renewcommand{\baselinestretch}{1.05} % a touch more line spacing
\setlength\parindent{1em}

% ----------- Typography -----------
\usepackage{microtype} % better spacing & justification


% ----------- Headings -----------
\usepackage{titlesec}
\titleformat{\section}{\large\bfseries}{\thesection}{1em}{}
\titleformat{\subsection}{\normalsize\bfseries}{\thesubsection}{0.75em}{}

% ----------- Lists -----------
\usepackage{enumitem}
\setlist{nosep} % compact itemize/enumerate

% ----------- Figures & Tables -----------
\usepackage{subcaption} % nicer subfigures
\usepackage{floatrow} % if you want aligned figure/table captions

% ----------- Footnotes -----------
\usepackage[hang,flushmargin]{footmisc} % neat footnotes

% ----------- Hyperlinks -----------
\usepackage[colorlinks=true,
    linkcolor=blue,
    citecolor=blue,
    urlcolor=blue]{hyperref}

% ----------- Column balance -----------
\usepackage{balance}

% ----------- Bibliography -----------
\usepackage[backend=biber,style=apa]{biblatex}
\DeclareLanguageMapping{dutch}{dutch-apa} % APA 7 in Dutch formatting
\addbibresource{main.bib}

% ----------- Acmart settings -----------
\settopmatter{printacmref=false}
\setcopyright{none}

% ----------- custom library -----------
% --------------------------------
% This file contains a custom set
% of commands to make writing
% LaTeX easier
% --------------------------------

% makes 0.05 inches of horizontal space
\newcommand{\vertspace}{\vspace{0.05in}}

% writes centered italic text 80% width of the page/column and gives top & bottom padding
\newcommand{\question}[1]{\vertspace\begin{center}\parbox{0.8\linewidth}{\centering\textit{#1}}\end{center}\vertspace}

% Writes a single paragraph of Lorum Ipsum text
\newcommand{\lorem}{Lorem ipsum dolor sit amet, consectetur adipiscing elit. Nullam eget tortor a urna ornare pellentesque. Integer sit amet purus nec sem iaculis euismod. Duis at ipsum eu libero pharetra egestas. Quisque eleifend odio velit, at sollicitudin metus dictum eu. Integer nec mi congue, gravida nibh sed, faucibus mauris. Sed vel ipsum lobortis felis gravida dignissim. Curabitur vestibulum turpis eu orci lacinia consectetur.}

%


% ----------- Authors and title -----------
\title{Contextual integrity}
\date{} % acmart usually overrides dates

\author{Matt ter Steege}
\date{Oktober 2025}
\email{m.j.ter.steege@students.uu.nl}
\affiliation{
    \institution{Universiteit Utrecht, 9932003}
    \city{Utrecht}
    \country{Nederland}
}

\begin{document}

    \maketitle

    \section*{Abstract}
    \lorem

    \section{Introductie}
    We leven in een tijd waarin zoveel mogelijk onderdelen van iemands leven aan het internet gekoppeld (kunnen) worden.
    Zo ook je eigen voordeur: de opkomst van zogenaamde videodeurbellen, zoals deurbellen van Ring of Eufy, is een steeds bekender gezicht in de wijken van Nederland.
    Het plus- (en tevens ook min-)punt van deze producten is dat elke (verdachte) beweging die deurbel detecteert, wordt opgenomen en doorgestuurd naar de eigenaar.
    Mogelijke inbrekers worden afgeschrikt door het idee dat ze op video staan bij een inbraakpoging en dat zorgt bij veel mensen voor een veilig gevoel, maar dit heeft ook een keerzijde.
    De postbode die je krantje komt bezorgen, maar ook voorbijrijdende auto’s, buren die een ommetje maken of kinderen die langsfietsen worden ook opgenomen, terwijl dit niet de doelgroep is waarvoor (of waartegen) deze deurbel ontworpen is.
    Dit roept de vraag op:

    \question{Hoe beïnvloedt het constant filmen van slimme deurbellen de privacy van buren en voorbijgangers?}

    Deze vraag sluit nauw aan bij het concept contextual integrity van Helen Nissenbaum, waarin iemand zo goed mogelijk in zijn of haar persoonlijke vrijheid gelaten wordt en data alleen in een passende context gedeeld mag worden.
    Videodeurbellen doorbreken deze verwachte informatiestromen: waar voorbijgangers normaal anoniem door een straat konden lopen, worden zij nu onbewust onderdeel van een digitaal surveillancesysteem.

    \subsection{Theoretisch kader}
    % CHECK: Contextual integrity uitleggen.
    % CHECK: Benoem hoe dit verschilt van het klassieke idee van privacy (bijv. “control over information”).
    % CHECK: Breng het naar jouw onderwerp: de context van de stoep voor een huis → normaal geen registratie, maar met een deurbelcamera wel. Hierdoor raakt de normale informatiestroom verstoord.

    \parencite{nissenbaum2009privacy} Schreef al over een door haar ontwikkeld privacy theorie: \textbf{Contextual integrity}.
    Dit schreef zij in haar boek \textit{Privacy In Context: Technology, Policy, and the Integrity of Social Life.}

    \vertspace
    \begin{itemize}[leftmargin = *,listparindent =1cm]
        \item[-] Privacy wordt gewaarborgd door passende informatiestromen.
        \item[-] Passende informatiestromen zijn stromen die voldoen aan contextuele informatienormen.
        \item[-] Contextuele informatienormen verwijzen naar vijf onafhankelijke parameters: betrokkene, afzender, ontvanger, informatietype en transmissie-principe.
        \item[-] Concepties van privacy zijn gebaseerd op ethische overwegingen die in de loop der tijd evolueren.
    \end{itemize}
    \vertspace

    Nissenbaum stelt dat privacy en wat acceptabel is om te delen af hangt van de situatie waarin men op dat moment leeft.
    Een voorbeeld hiervan is dat het (doorgaans vaak) niet gewenst is om je medische dossier met Jan en alleman te delen, echter met een dokter of huisarts is dit natuurlijk wel wenselijk.
    Hier komt contextuele integriteit goed naar boven.
    Want gebaseerd op de situatie deel je (of wil je) wel of niet bepaalde data met bepaalde entiteiten en deze entiteiten deze data ook niet doorgeven aan andere waarvoor de data niet nodig is.

    Dit wijkt af van het "traditionele" denkbeeld, oftwel \textit{control over information} waarin een individu zelf zijn data beheert en kiest of data wel of niet gedeeld wordt.
    Dit is veel meer individu-gecentreerd en contextuele integriteit is meer (je raadt het al) context-gecentreerd.

    \subsection{Relevantie}
    Videodeurbellen zijn dus nauw verbonden met het concept contextual integrity.
    Voor eigen veiligheid (of gemoedsrust) schaffen steeds meer mensen een videodeurbel aan, dit gaat echter ten koste van de privacy van voorbijgangers, buren en andere die toevallig langs een huis met een videodeurbel lopen.
    Daarom wordt in dit onderzoek gekeken naar of de waarde in veiligheidsgevoel opweegt tegen het ongevraagd (en passief) filmen van voorbijgangers en dergelijke.

    \section{Methode}
    Dit onderzoek is een kwantitatief onderzoek om de vragen rondom de contextuele integriteit en privacy en Slimme (video) deurbellen te beantwoorden.
    Hiervoor is uitsluitend literatuuronderzoek gedaan.
    Er zijn veel verschillende bronnen geraadpleegd, grotendeels via Google Scholar gedaan.
    Hierbij zijn verschillende zoektermen gebruikt zoals: Slimme deurbel, Ring (video)deurbel, Smart doorbell, Privacy video doorbell.


    \subsection{Inclusie- en exclusiecriteria}
    Contextual integrity uitleggen.

    \section{Beschouwing van literatuur}
    Hier ga je echt de gevonden artikelen samenbrengen in een doorlopend verhaal.
    Opdelen in subthema’s:

    \begin{itemize}[leftmargin = *,listparindent =1cm]
        \item[-] Bewegingsdetectie, cloudopslag, delen met politie.
        \item[-] Studies over hoe vaak mensen ongewild gefilmd worden; klachten; gevoelens van surveillance.
        \item[-] Wat bewoners ervaren: afschrikking, bewijs bij criminaliteit. Literatuur die laat zien of dit effect groot/klein is.
        \item[-] Hoe de informatiestromen door de deurbel afwijken van de ‘normale sociale verwachtingen’: een toevallige voorbijganger verwacht niet dat zijn route naar de supermarkt opgenomen en bewaard wordt.
        \item[-] Burenruzies, wantrouwen, normalisering van surveillance in de publieke ruimte.
    \end{itemize}

    Hier beschouw je de literatuur: niet alleen “wat zegt studie X”, maar ook: hoe hangen deze bevindingen samen, waar spreken ze elkaar tegen, en wat valt jou op.

    \section{Conclusie}
    Duidelijk antwoord op je onderzoeksvraag.
    Trek de lijn terug naar contextual integrity: veiligheid en privacy staan niet los van elkaar, maar de balans verschuift zodra technologie te veel buiten de intended context gaat. (max 1 pagina)

    \section{Voorbeeldcitatie}
    Lamport beschreef het gebruik van als documentopmaaksysteem al in de jaren negentig \parencite{tabassum2023exploring}.
    Ook Knuths werk over blijft invloedrijk. \parencite{chaudhari2020smart}
    blablalbalbalba \parencite{kelly2023ring}

    \printbibliography

    \balance % balances last-page columns
\end{document}
